 
%
\documentclass{llncs}

\usepackage{graphicx}      % include this line if your document contains figures
%\usepackage{natbib}        % required for bibliography


\usepackage{amsmath} 

\usepackage{amssymb} 
%\newtheorem{thm}{Theorem}
 
\newtheorem{ex}{Exercise}

\newtheorem{expl}{Example}


\newtheorem{prop}{Proposition}

\newtheorem{thm}{Theorem}


\begin{document}

\title{Springer LNCS Example Paper}
%

\author{Ivar Ekeland\inst{1} \and Roger Temam\inst{2} Jeffrey Dean \and David Grove \and Craig Chambers \and Kim~B.~Bruce \and Elsa Bertino} % \authorrunning{Ivar Ekeland et al.} % abbreviated author list (for running head) % %%%% list of authors for the TOC (use if author list has to be modified) 
\tocauthor{Ivar Ekeland, Roger Temam, Jeffrey Dean, David Grove, Craig Chambers, Kim B. Bruce, and Elisa Bertino} 
 \institute{Princeton University, Princeton NJ 08544, USA,\\ \email{I.Ekeland@princeton.edu},\\ WWW home page: bb
 \and Universit\'e de Paris-Sud, Laboratoire d'Analyse Num\'erique,\\
F-91405 Orsay Cedex, France}
%



\maketitle              % typeset the title of the contribution

\begin{abstract}
The abstract should summarize the contents of the paper using at least 70 and at most 150 words. It will be set in 9-point font size and be inset 1.0 cm from the right and left margins.
There will be two blank lines before and after the Abstract. \dots \keywords{computational geometry, graph theory, Hamilton cycles} \end{abstract}  

\section{Hamilton}
%
\subsection{Autonomous Systems}
%
In this section, we will consider the case when the Hamiltonian $H(x)$ is autonomous. For the sake of simplicity, we shall also assume that it is $C^{1}$.
%
\subsubsection{The General Case: Nontriviality.} % We assume that $H$ is $\left(A_{\infty},B_{\infty}\right)$-sub\-qua\-dra\-tic at infinity, for some constant symmetric matrices $A_{\infty}$ and $B_{\infty}$, with $B_{\infty}-A_{\infty}$ positive definite. Set:
\begin{eqnarray}
\gamma :&=&{\rm smallest\ eigenvalue\ of}\ \ B_{\infty} - A_{\infty} \\
   \lambda : &=& {\rm largest\ negative\ eigenvalue\ of}\ \
   J \frac{d}{dt} +A_{\infty}\ .
\end{eqnarray}

\begin{proposition}
Assume $H'(0)=0$ and $ H(0)=0$. Set:
\begin{equation}
   \delta := \liminf_{x\to 0} 2 N (x) \left\|x\right\|^{-2}\ .
   \label{eq:one}
\end{equation}

If $\gamma < - \lambda < \delta$,
the solution $\overline{u}$ is non-zero:
\begin{equation}
   \overline{x} (t) \ne 0\ \ \ \forall t\ .
\end{equation}
\end{proposition}
%
\begin{proof}
Condition (\ref{eq:one}) means that, for every $\delta ' > \delta$, there is some $\varepsilon > 0$ such that \begin{equation}
   \left\|x\right\| \le \varepsilon \Rightarrow N (x) \le
   \frac{\delta '}{2} \left\|x\right\|^{2}\ .
\end{equation}
\end{proof}

%

\begin{lemma}
Assume that $H$ is $C^{2}$ on $\bbbr^{2n} \setminus \{ 0\}$ and that $H'' (x)$ is non-de\-gen\-er\-ate for any $x\ne 0$. Then any local minimizer $\widetilde{x}$ of $\psi$ has minimal period $T$.
\end{lemma}
%
\begin{proof}
We know that $\widetilde{x}$, or
$\widetilde{x} + \xi$ for some constant $\xi \in \bbbr^{2n}$, is a $T$-periodic solution of the Hamiltonian system:
\begin{equation}
   \dot{x} = JH' (x)\ .
\end{equation}
FF.\qed \end{proof} 

\begin{table}
\caption{This is the example table taken out of {\it The \TeX{}book,} p.\,246} \begin{center} \begin{tabular}{r@{\quad}rl} \hline \multicolumn{1}{l}{\rule{0pt}{12pt}
                    Year}&\multicolumn{2}{l}{World population}\\[2pt] \hline\rule{0pt}{12pt}
8000 B.C.  &     5,000,000& \\
   50 A.D.  &   200,000,000& \\
1650 A.D.  &   500,000,000& \\
1945 A.D.  & 2,300,000,000& \\
1980 A.D.  & 4,400,000,000& \\[2pt]
\hline
\end{tabular}
\end{center}
\end{table}
%
\begin{theorem} [Ghoussoub-Preiss]\label{ghou:pre}
Assume $H(t,x)$ is
\qed
\end{theorem}
%
\begin{example} [{{\rm External forcing}}] Consider the system:
\begin{equation}
   \dot{x} = JH' (x) + f(t)
\end{equation}
\end{example}
%
\begin{definition}
Let $A_{\infty} (t)$ and $B_{\infty} (t)$ be symmetric
\end{definition}
%

\paragraph{Notes and Comments.}
The first results on subharmonics were


%
% ---- Bibliography ----
%
\begin{thebibliography}{5}
%
\bibitem {clar:eke}
Clarke, F., Ekeland, I.:
Nonlinear oscillations and
boundary-value problems for Hamiltonian systems.
Arch. Rat. Mech. Anal. 78, 315--333 (1982)

\bibitem {clar:eke:2}
Clarke, F., Ekeland, I.:
Solutions p\'{e}riodiques, du
p\'{e}riode donn\'{e}e, des \'{e}quations hamiltoniennes.
Note CRAS Paris 287, 1013--1015 (1978)

\bibitem {mich:tar}
Michalek, R., Tarantello, G.:
Subharmonic solutions with prescribed minimal period for nonautonomous Hamiltonian systems.
J. Diff. Eq. 72, 28--55 (1988)

\bibitem {tar}
Tarantello, G.:
Subharmonic solutions for Hamiltonian
systems via a $\bbbz_{p}$ pseudoindex theory.
Annali di Matematica Pura (to appear)

\bibitem {rab}
Rabinowitz, P.:
On subharmonic solutions of a Hamiltonian system.
Comm. Pure Appl. Math. 33, 609--633 (1980)

\end{thebibliography}

\end{document}